\documentclass{article}

\usepackage[utf8]{inputenc}
\usepackage[spanish]{babel}
\usepackage{amsmath}
\begin{document}
\title{Ecuacion en diferencias de primer orden}
1.Escribe la fórmula de la solucion a la ecuacion en diferencias lineal homogenea de primer orden con condicion inicial:
$$x_{n-1}=ax_n,$$
$$x_0=C.$$

Observemos un patron calculando los primeros $x_1,x_2 y x_3$
\begin{align*}
X_1&=aX_0=aC\\
x_2&=ax_1=a(aC)1=a^2C\\
x_3&=ax_2=a(a^2C)=a^3C
\end{align*}
Observando lo anterior, sospechamos que la solucion es: $$x_n=a^nC$$

Para probarlo utilizamos induccion, veamos que se cumple para n=1 $$a^1C=ax_0=x_1$$
Ahora supongamos que se sumple para $k$ y probemos que esto implica que se cumple para $k+1$

\end{document}